\documentclass[a4paper]{article}
\usepackage[utf8]{inputenc}
\usepackage[margin=2cm]{geometry}
\usepackage{tikz-uml}

\begin{document}
  \begin{titlepage}
    \vfill
      \begin{center}
        {\large \textbf{UNIVERSIDADE FEDERAL DOS VALES DO JEQUITINHONHA E MUCURI}} \\[0.5cm]
        {\large \textbf{SISTEMAS DE INFORMAÇÃO}} \\[4cm]
        
        {\large \textbf{Gabreil Morais}}\\[1cm]
        {\large \textbf{Tales Félix}}\\[4cm]
    
        {\Large Padrão de Projeto - Model View Controller  X Observer }\\[4cm]
    
        \hspace{.45\textwidth} %posiciona a minipage
        \begin{minipage}{.5\textwidth}
        \large
        Docente: Eduardo Pelli.\\[8cm]
    \end{minipage}
    {\large Diamantina, 18 de Outubro de 2020}
    \end{center}
  \end{titlepage}


  \newpage
  
    \section{Model View Controller - Introdução}
    O MVC é um padrão de projeto voltado para arquitetura de software, comumente usado na
    web. Seu principal objetivo é a separação de funcionalidades em três camadas, model, view e controller.\\
    O model é responsalvel pela regra de negócio, gerenciamento de dados etc. já o view fica responsavel
    pela parte de visualização do usuário, como gráficos e tabelas de diferentes formas. Para fezer a comunicação
    entre essas duas camadas usa se o controller, que fica responsavel pela comunicação, respostas e requisições
    entre o model e view.
    \subsection{Problematização de um projeto em MVC}
    Um colecionador precisa de um sistema para guardar informações de seus objetos.
    O sistema deverá fazer registro do nome e descrição geral de cada pertence.
    Ao ser solicitado pelo usuário o sistema deverá retonar uma lista contendo
    todos objetos cadastrados.
    
    
    
    \subsection{Diagrama de Casos de Uso}

    \begin{tikzpicture}

      \begin{umlsystem}{Casos de uso}
       %Casos de uso
      \umlusecase[x=-5,y=8]{cadastrar Objeto}
      \umlusecase[x=-5,y=6]{Altera Objeto}
      \umlusecase[x=-5,y=4]{Excluir Objeto}
      \umlusecase[x=-5,y=2]{Listar Objetos}

    
      
      \end{umlsystem}
      
      
      % \umlextend  {usecase-3}{usecase-6}
      
      
      %atores
      \umlactor[x=-12,y=5]{Usuario}
      
      %include
      %\umlinclude{usecase-1}{usecase-2}
      
      %extend
      %\umlextend{usecase-5}{usecase-4}

      %Ator relacionamentos
      \umlassoc{Usuario}{usecase-1}
      \umlassoc{Usuario}{usecase-2}
      \umlassoc{Usuario}{usecase-3}
      \umlassoc{Usuario}{usecase-4}

      
      
  \end{tikzpicture}\\


 \subsection{Fluxo de Eventos}
    \subsubsection{Cadastrar Objeto}
      O usuário Deverá preencher os campos, "Nome" e "Descrição" e sobmeter clicando no botão Enviar.\\
      Aparecerá uma mensagem de sucesso, caso contrário mostrarar um erro e o usuário terá que repetir a operação.\\

      \subsubsection{Listar Objetos}
    Após clicar em "Listar" aparecerar uma lista de elementos.\\
    Caso não tenha nenum matrial cadrastado mostrarar a lista vazia.\\
    
    \subsubsection{Alterar Objetos}
    O usuário Deverá preencher os campos, "Nome", "Descrição" e "ID" e submeter\\
    Se os ID do objeto que deseja alterar não estiver devidadamente preenchido mostrarar uma mensagem de erro.\\
    Caso contrário aparecerá uma mensagem de sucesso\\

    \subsubsection{Excluir Objeto}
    O usuário Deverá preencher "ID" e submeter\\
    Se os ID do objeto que deseja excluir não estiver devidadamente preenchido mostrarar uma mensagem de erro.\\
    Caso contrário aparecerá uma mensagem de sucesso\\
  
  
  \subsection{Classes Lista de Coleção}
  \begin{tikzpicture} 
    %\begin{umlpackage}{p}
      
      \begin{umlpackage}{Lista de Colecao} 

        \umlclass[x=-10, y=0]{Registro}{ 
          \umlstatic{- colecao : Colecao[]}\\
        }{
          + incluirColecao(Colecao: colecao): bool\\
          + EnviarListaColecao(): String\\
          + Excluir(String: id): bool\\
          + Alterar(id: String, nome: String, descricao: String):bool\\
        }
        
      \umlclass[x=0, y=-7]{Colecao}{
        - nome : String \\
        - descricao : String \\
        - id : String \\}{
        + Colecao(nome: String, descricoa: String)\\
        + getNome(): String\\
        + setNome(nome: String): void\\
        + getDescricao(): String\\
        + setDecriscao(descricao: String): void\\
        + getId(): String\\
        + setId(id: String): void\\
        + toString(): String\\}

        \umlclass[x=0, y=0]{Arquivo}{
          \umlstatic{- escrita: FileWriter}\\
          \umlstatic{- parser: JSONParser}\\
          \umlstatic{- scan: Scanner}\\
          \umlstatic{- gson: Gson}\\
        }{
          \umlstatic{+ puxarDados(): void}\\
          \umlstatic{+ escrever(): void}\\
          \umlstatic{+ liparArquivo(): void}\\
          \umlstatic{+ enviarParaEscrita(): void}
        }
      
      \umlclass[x=-10,y=-5]{Routes}{
        }{
          Salvar(): bool\\
          Listar(): String\\
          Alterar(id: String, nome: String, descricao: String): boll\\
          Excluir(id: String): boll\\
          }

      \umlclass[x=-10, y=-10]{Index}{
        - nome: String\\
        - descricao: String\\
        - id: String\\
        - IdEccluir: String\\
        - IdAterar: Strig\\
        - nomeAlterar: String\\
        - descricaoAlterar: String\\
      }{
        + EventEnviar()\\
        + EventAterar()\\
        + EventExcluir()\\
        + EventListar()\\
      }
          
        \end{umlpackage}
\end{tikzpicture}\\[0.5cm]

    
    \subsection{Diagrama de Classes Lista de Coleções}
    \begin{tikzpicture} 
      \begin{umlpackage}{MVC}
        
        \begin{umlpackage}[x=-10, y=-2]{view} 
  
            \umlclass[x=0, y=0, scale=0.5]{Index}{
          - nome: String\\
          - descricao: String\\
          - id: String\\
          - IdEccluir: String\\
          - IdAterar: Strig\\
          - nomeAlterar: String\\
          - descricaoAlterar: String\\
        }{
          + EventEnviar()\\
          + EventAterar()\\
          + EventExcluir()\\
          + EventListar()\\
        }
              
            \end{umlpackage}
  
      \begin{umlpackage}{model} 
        \umlclass[x=0, y=0, scale=0.5]{Registro}{ 
          \umlstatic{- colecao : Colecao[]}\\
        }{
          + incluirColecao(Colecao: colecao): bool\\
          + EnviarListaColecao(): String\\
          + Excluir(String: id): bool\\
          + Alterar(id: String, nome: String, descricao: String):bool\\
        }
        \umlclass[x=0, y=-3, scale=0.5]{Colecao}{
          - nome : String \\
          - descricao : String \\
          - id : String \\}{
          + Colecao(nome: String, descricoa: String)\\
          + getNome(): String\\
          + setNome(nome: String): void\\
          + getDescricao(): String\\
          + setDecriscao(descricao: String): void\\
          + getId(): String\\
          + setId(id: String): void\\
          + toString(): String\\}
  
          \umlclass[x=-4, y=-2, scale=0.5]{Arquivo}{
            \umlstatic{- escrita: FileWriter}\\
            \umlstatic{- parser: JSONParser}\\
            \umlstatic{- scan: Scanner}\\
            \umlstatic{- gson: Gson}\\
          }{
            \umlstatic{+ puxarDados(): void}\\
            \umlstatic{+ escrever(): void}\\
            \umlstatic{+ liparArquivo(): void}\\
            \umlstatic{+ enviarParaEscrita(): void}
          }
      \end{umlpackage}
      \begin{umlpackage}[x=-10,y=-7]{controller} 
  
        \umlclass[x=0,y=0, scale=0.5]{Routes}{
          }{
            Salvar(): bool\\
            Listar(): String\\
            Alterar(id: String, nome: String, descricao: String): boll\\
            Excluir(id: String): boll\\
            }  
          \end{umlpackage} 
  
          
        \end{umlpackage}
        \umlcompo{Registro}{Colecao}
        \umlassoc{Registro}{Colecao}
  
        \umlimport{controller}{model}

        \umlimport{view}{controller}
 
        \umluniassoc{Arquivo}{Registro} 
          \end{tikzpicture}\\[0.5cm]



  \subsection{Diagrama de classe MVC}
  \begin{tikzpicture} 
    \begin{umlpackage}{MVC}
      
      \begin{umlpackage}{view} 

        \umlclass[x=0, y=-5]{view}{ 
          }{
            
            }
            
          \end{umlpackage}

    \begin{umlpackage}{model} 
      \umlclass[x=10, y=-5]{Model}{ 
       
      }{
      }
    \end{umlpackage}
    \begin{umlpackage}{controller} 

      \umlclass[x=5, y=0]{Controller}{ 
      
      }{

      }
          
        \end{umlpackage} 

        
      \end{umlpackage}
      \umlimport{controller}{model}
      \umlimport{view}{controller}
        \end{tikzpicture}\\[0.5cm]

    \section{Observer}

    \subsection{Objetivo}

    Observer é um padrão de projeto comportamental que permite que um objeto notifique outros objetos sobre a mudança em seu estado. O padrão Observer fornece uma maneira de assinar e cancelar a assinatura desses eventos para qualquer objeto que implemente uma interface de assinante. Sendo assim o observar é um padrão de projeto que tem como princípio a ligação leve.
	Sendo assim, o princípio da ligação leve procurar objetos levemente ligados um com o outro, para que eles saibam muito pouco de cada um, sendo assim interagindo normalmente.


    \subsection{Diagrama Observer}

    O diagrama do observer ele tem duas classes, uma subject e uma observer, sendo a subject como a observado, na qual a observer recebe as notificações , sendo assim notificando os outros objetos que o objeto observado mudou de estado\\\\

    \begin{tikzpicture} 
      \begin{umlpackage}{observer}
        
  
          \umlclass[x=1, y=3]{Observer}{ 
            }{
            + notify()
              }
              
  
        \umlclass[x=7, y=3]{Subject}{ 
         + observerCollection
        }{
          + registerObsertver(observer)\\
          + unregisterObserver(observer)\\
          + notifyObservers()
        }


  
        \umlclass[x=0, y=0]{ConcreteObserverA}{ 
        
        }{
          + notify()
        }

        \umlclass[x=5, y=0]{ConcreteObserverB}{ 
        
          }{
            + notify()
          }
            
          \end{umlpackage} 
  
          \umlinherit{ConcreteObserverB}{Observer}
          \umlinherit{ConcreteObserverA}{Observer}
          \umlaggreg{Subject}{Observer}


          \end{tikzpicture}\\[0.5cm]

          \subsection{Diagrama de classes}

          Este projeto é uma implementação do padrão de projeto observer, em um 
          cenário de um resgate do samu, e a chegada no hospital, sendo assim ele
          designa depois de um leve diagnóstico a que local do hospital, o paciente s
          erá encaminhado.\\
          Devido a esse cenário temos 2 classes , a Atendente e o Samu. Na qual a 
          atendente e uma classe observadora da classe samu, contendo nela três métodos
          diaguinosticar1():void, diaguinosticar2():void, diaguinosticar3():void e 
          update que é um método que foi declarado na interface observer. \\
          Tendo também a classe Samu na qual tem os atributos acao String , 
          diagnóstico String, e paciente String e sendo os métodos paciente():void , diagnostico():void e mudaEstado():void . Devido a esses métodos, o diagnóstico passa no método, faz uma comparação passando para a variável ação que por sua vez se chama this.mudaEstado, sendo assim notificando seu observador, que por sua vez na sua função update compara a ação e chama um método do atendente .
          Prós x contras.\\
          Princípio  do aberto ou fechado pode-se introduzir novas classes 
          assinantes a qualquer momento. Devido que um objeto observado pode 
          receber novos observadores a qualquer momento. Sendo assim, gerando 
          conexões com objetos a qualquer momento. \\
          O ponto negativo, assinantes são notificados de forma. 
          Se é emitido um sinal que muda o estado do objeto observado, então 
          todos os observadores vão reagir a essa mudança de estado.


        \subsection{Diagrama de Classes Observer}
        \begin{tikzpicture} 
          \begin{umlpackage}{observer}
            
      
              \umlclass[x=0, y=0]{Teste}{ 
                }{

                  }
                  
      
            \umlclass[x=4, y=2]{Atendente}{ 
            
            }{
              + Diagnostico(): void\\
              + Diagnostico1(): void\\
              + Diagnostico2(): void
            }
    
    
      
            \umlclass[x=10, y=5]{Observable}{
              
            
            }{
              + setChange(): void\\
              + nofyObserver(): void
            }
    
            \umlclass[x=10, y=0]{Samu}{ 
                - acao: String\\
                - paciente: String
                diagnostico: String\\
              }{
                - Paciente(): void\\
                - Diagnostico(): void\\
                + mudarEstacao(): void
              }

              \umlclass[x=4, y=5]{Observer}{ 

              }{
                + Update(args0: int, agr1: int): void
              }
                
              \end{umlpackage} 
      
              \umlinherit{Teste}{Atendente}
              \umlinherit{Teste}{Samu}
              \umlinherit{Samu}{Observable}
              \umlimpl{Atendente}{Observer}
    
              
              \end{tikzpicture}\\[0.5cm]

  \section{Comparação entre MVC e Observer}

  Observer é um padrão comportamental , no qual tem a maior função notificar outros 
  objetos no qual eles ficam de observadores, para que possam mudar de estado. 
  No entanto, o MVC é um projeto de arquitetura de software que tem o intuito de 
  dividir responsabilidade do projeto em três camadas, sendo elas model, view, controller. 
  No entanto um módulo do mvc pode usar o padrão observer, sendo ele o model, que 
  por sua vez pode fazer o uso do padrão Observer que separa a visão do estado de 
  um objeto do próprio objeto, na qual permite que o objeto forneça visões alternativas 
  mantendo os objetos interessados constantemente, informados sobre suas mudanças de estado.

  \section{Conclusão}
     

  Tendo em vista que o uso de padrões de projeto é essencial para o desenvolvimento 
  de alguma aplicação que pode vir a ser desenvolvida, ela entra também no uso de 
  boas práticas, no qual permite ao desenvolvedor ter um código bem organizado e 
  com grande chance de realização e manutenção. \\
  Essas técnicas têm grande importância no mundo da programação, na qual tem a 
  necessidade de bom entendimento e de bom uso , para que se possa ter códigos 
  bem estruturados e de fácil entendimento.


  \newpage
  \begin{thebibliography}{4}
    \bibitem{DEVMEDIA}DEVMEDIA.\textbf{Padrão de Projeto Observer em Java.}
    Recuperado em 18 de dezembro de 2020,
   https://www.devmedia.com.br/padrao-de-projeto-observer-em-java/26163

   \bibitem{DEVMEDIA}DEVMEDIA.\textbf{O Padrão de Projeto Observer.} 
   Recuperado em 18 de dezembro de 2020,
   https://www.devmedia.com.br/o-padrao-de-projeto-observer/22861
   \bibitem{Linha Codigo}DEVMEDIA.\textbf{Abordando a arquitetura MVC, e Design Patterns: Observer, Composite, Strategy.
   }
   Recuperado em 19 de dezembro de 2020,
   https://www.devmedia.com.br/o-padrao-de-projeto-observer/22861
   
   
   
    \end{thebibliography}

     
  \end{document}