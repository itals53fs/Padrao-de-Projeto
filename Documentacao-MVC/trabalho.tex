\documentclass[a4paper]{article}
\usepackage[utf8]{inputenc}
\usepackage[margin=2cm]{geometry}
\usepackage{tikz-uml}

\begin{document}
  \begin{titlepage}
    \vfill
      \begin{center}
        {\large \textbf{UNIVERSIDADE FEDERAL DOS VALES DO JEQUITINHONHA E MUCURI}} \\[0.5cm]
        {\large \textbf{SISTEMAS DE INFORMAÇÃO}} \\[4cm]
    
        {\large \textbf{Tales Félix}}\\[4cm]
    
        {\Large Padrão de Projeto - MVC }\\[4cm]
    
        \hspace{.45\textwidth} %posiciona a minipage
        \begin{minipage}{.5\textwidth}
        \large
        Docente: Eduardo Pelli.\\[8cm]
    \end{minipage}
    {\large Diamantina, 18 de Outubro de 2020}
    \end{center}
  \end{titlepage}


  \newpage
    \section{Introdução}
    Um colecionador precisa de um sistema para guardar informações de seus objetos.
    O sistema deverá fazer registro do nome e descrição geral de cada pertence.
    Ao ser solicitado pelo usuário o sistema deverá retonar uma lista contendo
    todos objetos cadastrados.
    
    
    
    \section{Diagrama de Casos de Uso}

    \begin{tikzpicture}

      \begin{umlsystem}{Casos de uso}
       %Casos de uso
      \umlusecase[x=-5,y=8]{cadastrar Objeto}
      \umlusecase[x=-5,y=6]{Altera Objeto}
      \umlusecase[x=-5,y=4]{Excluir Objeto}
      \umlusecase[x=-5,y=2]{Listar Objetos}

    
      
      \end{umlsystem}
      
      
      % \umlextend  {usecase-3}{usecase-6}
      
      
      %atores
      \umlactor[x=-12,y=5]{Usuario}
      
      %include
      %\umlinclude{usecase-1}{usecase-2}
      
      %extend
      %\umlextend{usecase-5}{usecase-4}

      %Ator relacionamentos
      \umlassoc{Usuario}{usecase-1}
      \umlassoc{Usuario}{usecase-2}
      \umlassoc{Usuario}{usecase-3}
      \umlassoc{Usuario}{usecase-4}

      
      
  \end{tikzpicture}\\

      O diagrama de casos de uso corresponde a uma visão externa do sistema
      e representa graficamente os atores, os casos de uso, e os relacionamentos
      entre estes elementos. Ele tem como objetivo ilustrar em um nível alto de
      abstração quais elementos externos interagem com que funcionalidades do sistema,
      ou seja, a finalidade de um diagrama de caso de uso é apresentar um tipo de diagrama
      de contexto que apresenta os elementos externos de um sistema e as maneiras segundo
      as quais eles as utilizam.

 \section{Fluxo de Eventos}
    \subsection{Cadastrar Objeto}
      O usuário Deverá preencher os campos, "Nome" e "Descrição" e sobmeter clicando no botão Enviar.\\
      Aparecerá uma mensagem de sucesso, caso contrário mostrarar um erro e o usuário terá que repetir a operação.\\

    \subsection{Listar Objetos}
    Após clicar em "Listar" aparecerar uma lista de elementos.\\
    Caso não tenha nenum matrial cadrastado mostrarar a lista vazia.\\
    
    \subsection{Alterar Objetos}
    O usuário Deverá preencher os campos, "Nome", "Descrição" e "ID" e submeter\\
    Se os ID do objeto que deseja alterar não estiver devidadamente preenchido mostrarar uma mensagem de erro.\\
    Caso contrário aparecerá uma mensagem de sucesso\\

    \subsection{Excluir Objeto}
    O usuário Deverá preencher "ID" e submeter\\
    Se os ID do objeto que deseja excluir não estiver devidadamente preenchido mostrarar uma mensagem de erro.\\
    Caso contrário aparecerá uma mensagem de sucesso\\
  
  
  \section{Classes}
  \begin{tikzpicture} 
    %\begin{umlpackage}{p}
      
      \begin{umlpackage}{Lista de Colecao} 


        \umlclass[x=-10, y=0]{Registro}{ 
          - colecao : Colecao[] \\
        }{
          + incluirColecao(Colecao: colecao): bool\\
          + EnviarListaColecao(): String\\
          + Excluir(String: id): bool\\
          + Alterar(id: String, nome: String, descricao: String):bool\\
        }

        
      \umlclass[x=0, y=-10]{Colecao}{
        - nome : String \\
        - descricao : String \\
        - id : String \\}{
        + Colecao(nome: String, descricoa: String)\\
        + getNome(): String\\
        + setNome(nome: String): void\\
        + getDescricao(): String\\
        + setDecriscao(descricao: String): void\\
        + getId(): String\\
        + setId(id: String): void\\
        + toString(): String\\}

        \umlclass[x=0, y=0]{Arquivo}{
        - escrita: FileWriter\\
        - parser: JSONParser\\
        - scan: Scanner\\
        - gson: Gson\\
      }{
        + readFile(): String\\
        + puxarDados(): void\\
        + escrever(): void\\
        + liparArquivo(): void\\
        + enviarParaEscrita(): void
      }

      
      \umlclass[x=-10,y=-5]{Routes}{
        }{
          Salvar(): bool\\
          Listar(): String\\
          Alterar(id: String, nome: String, descricao: String): boll\\
          Excluir(id: String): boll\\
          }

      \umlclass[x=-10, y=-10]{Index}{
        - nome: String\\
        - descricao: String\\
        - id: String\\
        - IdEccluir: String\\
        - IdAterar: Strig\\
        - nomeAlterar: String\\
        - descricaoAlterar: String\\
      }{
        + EventEnviar()\\
        + EventAterar()\\
        + EventExcluir()\\
        + EventListar()\\
      }
      \umlclass[x=0, y=0]{Arquivo}{
        - escrita: FileWriter\\
        - parser: JSONParser\\
        - scan: Scanner\\
        - gson: Gson\\
      }{
        + readFile(): String\\
        + puxarDados(): void\\
        + escrever(): void\\
        + liparArquivo(): void\\
        + enviarParaEscrita(): void
      }
          
        \end{umlpackage}
\end{tikzpicture}\\[0.5cm]



\section{Diagrama de Sequência}
    
    \section{Diagrama de Classes Lista de Coleções}
    \begin{tikzpicture} 
      %\begin{umlpackage}{p}
        
        \begin{umlpackage}{Lista de Colecao}
  
          \umlclass[x=-10, y=0]{Registro}{ 
          - colecao : Colecao[] \\
        }{
          + incluirColecao(Colecao: colecao): bool\\
          + EnviarListaColecao(): String\\
          + Excluir(String: id): bool\\
          + Alterar(id: String, nome: String, descricao: String):bool\\
        }
        
      \umlclass[x=0, y=-7]{Colecao}{
        - nome : String \\
        - descricao : String \\
        - id : String \\}{
        + Colecao(nome: String, descricoa: String)\\
        + getNome(): String\\
        + setNome(nome: String): void\\
        + getDescricao(): String\\
        + setDecriscao(descricao: String): void\\
        + getId(): String\\
        + setId(id: String): void\\
        + toString(): String\\}

        \umlclass[x=0, y=0]{Arquivo}{
          - escrita: FileWriter\\
          - parser: JSONParser\\
          - scan: Scanner\\
          - gson: Gson\\
        }{
          + readFile(): String\\
          + puxarDados(): void\\
          + escrever(): void\\
          + liparArquivo(): void\\
          + enviarParaEscrita(): void
        }
      
      \umlclass[x=-10,y=-5]{Routes}{
        }{
          Salvar(): bool\\
          Listar(): String\\
          Alterar(id: String, nome: String, descricao: String): boll\\
          Excluir(id: String): boll\\
          }

      \umlclass[x=-10, y=-10]{Index}{
        - nome: String\\
        - descricao: String\\
        - id: String\\
        - IdEccluir: String\\
        - IdAterar: Strig\\
        - nomeAlterar: String\\
        - descricaoAlterar: String\\
      }{
        + EventEnviar()\\
        + EventAterar()\\
        + EventExcluir()\\
        + EventListar()\\
      }

            
          \end{umlpackage} 
          \umlcompo{Registro}{Colecao}
          \umlassoc{Registro}{Colecao}
          \umlassoc{Routes}{Index}
          \umlassoc{Routes}{Registro}
          \umluniassoc{Arquivo}{Registro}

          \end{tikzpicture}\\[0.5cm]



  \section{Diagrama de classe MVC}
  \begin{tikzpicture} 
    \begin{umlpackage}{MVC}
      
      \begin{umlpackage}{view} 

        \umlclass[x=0, y=-5]{view}{ 
          }{
            
            }
            
          \end{umlpackage}

    \begin{umlpackage}{model} 
      \umlclass[x=10, y=-5]{Model}{ 
       
      }{
      }
    \end{umlpackage}
    \begin{umlpackage}{controller} 

      \umlclass[x=5, y=0]{Controller}{ 
      
      }{

      }
          
        \end{umlpackage} 

        
      \end{umlpackage}
      \umlimport{controller}{model}
      \umlimport{view}{controller} 
        \end{tikzpicture}\\[0.5cm]

  \section{Conclusão}


  \newpage
  \begin{thebibliography}{4}
    %\bibitem{DEVMEDIA}DEVMEDIA.\textbf{O Que é UML e Diagramas de Caso de Uso}: Introdução Prática à UML. 
    %Recuperado em 17 de outubro de 2020,
    %https://www.devmedia.com.br/o-que-e-uml-e-diagramas-de-caso-de-uso-introducao-pratica-a-uml/23408
    \end{thebibliography}
     
  \end{document}