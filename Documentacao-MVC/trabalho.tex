\documentclass[a4paper]{article}
\usepackage[utf8]{inputenc}
\usepackage[margin=2cm]{geometry}
\usepackage{tikz-uml}

\begin{document}
  \begin{titlepage}
    \vfill
      \begin{center}
        {\large \textbf{UNIVERSIDADE FEDERAL DOS VALES DO JEQUITINHONHA E MUCURI}} \\[0.5cm]
        {\large \textbf{SISTEMAS DE INFORMAÇÃO}} \\[4cm]
    
        {\large \textbf{Tales Félix}}\\[4cm]
    
        {\Large Padrão de Projeto - MVC }\\[4cm]
    
        \hspace{.45\textwidth} %posiciona a minipage
        \begin{minipage}{.5\textwidth}
        \large
        Docente: Eduardo Pelli.\\[8cm]
    \end{minipage}
    {\large Diamantina, 18 de Outubro de 2020}
    \end{center}
  \end{titlepage}


  \newpage
    \section{Introdução}
    Um colecionador precisa de um sistema para guardar informações de seus objetos.
    O sistema deverá fazer registro do nome e descrição geral de cada pertence.
    Ao ser solicitado pelo usuário o sistema deverá retonar uma lista contendo
    todos objetos cadastrados.
    
    
    
    \section{Diagrama de Casos de Uso}

    \begin{tikzpicture}

      \begin{umlsystem}{Casos de uso}
       %Casos de uso
      \umlusecase[x=-5,y=8]{cadastrar Objeto}
      \umlusecase[x=-5,y=6]{Altera Objeto}
      \umlusecase[x=-5,y=4]{Excluir Objeto}
      \umlusecase[x=-5,y=2]{Listar Objetos}

    
      
      \end{umlsystem}
      
      
      % \umlextend  {usecase-3}{usecase-6}
      
      
      %atores
      \umlactor[x=-12,y=5]{Usuario}
      
      %include
      %\umlinclude{usecase-1}{usecase-2}
      
      %extend
      %\umlextend{usecase-5}{usecase-4}

      %Ator relacionamentos
      \umlassoc{Usuario}{usecase-1}
      \umlassoc{Usuario}{usecase-2}
      \umlassoc{Usuario}{usecase-3}
      \umlassoc{Usuario}{usecase-4}

      
      
  \end{tikzpicture}\\

      O diagrama de casos de uso corresponde a uma visão externa do sistema
      e representa graficamente os atores, os casos de uso, e os relacionamentos
      entre estes elementos. Ele tem como objetivo ilustrar em um nível alto de
      abstração quais elementos externos interagem com que funcionalidades do sistema,
      ou seja, a finalidade de um diagrama de caso de uso é apresentar um tipo de diagrama
      de contexto que apresenta os elementos externos de um sistema e as maneiras segundo
      as quais eles as utilizam.

 \section{Fluxo de Eventos}
    \subsection{Cadastrar Objeto}
      O usuário Deverá preencher os campos, "Nome" e "Descrição" e sobmeter clicando no botão Enviar.\\
      Aparecerá uma mensagem de sucesso, caso contrário mostrarar um erro e o usuário terá que repetir a operação.\\

    \subsection{Listar Objetos}
    Após clicar em "Listar" aparecerar uma lista de elementos.\\
    Caso não tenha nenum matrial cadrastado mostrarar a lista vazia.\\
    
    \subsection{Alterar Objetos}
    O usuário Deverá preencher os campos, "Nome", "Descrição" e "ID" e submeter\\
    Se os ID do objeto que deseja alterar não estiver devidadamente preenchido mostrarar uma mensagem de erro.\\
    Caso contrário aparecerá uma mensagem de sucesso\\

    \subsection{Excluir Objeto}
    O usuário Deverá preencher "ID" e submeter\\
    Se os ID do objeto que deseja excluir não estiver devidadamente preenchido mostrarar uma mensagem de erro.\\
    Caso contrário aparecerá uma mensagem de sucesso\\
  
  
  \section{Classes}
  \begin{tikzpicture} 
    %\begin{umlpackage}{p}
      
      \begin{umlpackage}{Lista de Colecao} 

        \umlclass[x=-10, y=0]{Registro}{ 
        - colecao : Colecao[] \\
      }{
        + incluirColecao(Colecao: colecao): bool\\
        + EnviarListaColecao(): String\\
        + Excluir(String: id): bool\\
        + Alterar(id: String, nome: String, descricao: String):bool\\
      }
      
    \umlclass[x=0, y=-5]{Colecao}{
      - nome : String \\
      - descricao : String \\
      - id : String \\}{
      + Colecao(nome: String, descricoa: String)\\
      + getNome(): String\\
      + setNome(nome: String): void\\
      + getDescricao(): String\\
      + setDecriscao(descricao: String): void\\
      + getId(): String\\
      + setId(id: String): void\\
      + toString(): String\\}
    
    \umlclass[x=-10,y=-5]{Routes}{
      }{
        Salvar(): bool\\
        Listar(): String\\
        Alterar(id: String, nome: String, descricao: String): boll\\
        Excluir(id: String): boll\\
        }

    \umlclass[x=-10, y=-10]{Index}{}{}
          
        \end{umlpackage}
\end{tikzpicture}\\[0.5cm]



\section{Diagrama de Sequência}

\subsection{Diagrama de Sequência Cadrastar Colaborador}
    \begin{tikzpicture} 
      \begin{umlseqdiag} 
        \umlactor[]{ADM}
        \umlobject[stereo=entity]{Login}
        \umlobject[stereo=entity]{Validar}
        \umlobject[stereo=entity]{TelaSistema}
        \umlobject[stereo=entity]{ValidarCodigo}
        \begin{umlcall}[op={1},return=4]{ADM}{Login}
        \end{umlcall}
        \begin{umlcall}[op={2}, return=3]{Login}{Validar}
        \end{umlcall}
        \begin{umlcall}[op={5}, dt=10]{ADM}{TelaSistema}
        \end{umlcall}
        \begin{umlcall}[op={6}, return=8]{TelaSistema}{ValidarCodigo}
        \end{umlcall}
        \begin{umlcall}[op={9}]{TelaSistema}{ADM}
        \end{umlcall}
        \begin{umlcallself}[op={7}]{ValidarCodigo} 
        \end{umlcallself} 
      \end{umlseqdiag} 
    \end{tikzpicture}\\
    
    {\bf Descrição:}\\
    1. Logar()\\
    2. Validar Login\\
    3. Resposta validação\\
    4. Login Efetuado\\
    5. Formulário\\
    6. CadrastarColaborador()\\
    7. Validar Dados Inseridos\\
    8. Mostrar mensagem de sucesso ou mostrar erro\\
    9. Retonar Início\\
    
    
    No que se refere ao diagrama de seqüência, preocupa-se com a ordem temporal
    em que as mensagens são trocadas entre os objetos envolvidos em determinado
    processo, ou seja, quais condições devem ser satisfeitas e quais métodos devem
    ser disparados entre os objetos envolvidos e em que ordem durante um processo.
    Dessa forma, determinar a ordem em que os eventos ocorrem, as mensagens que são
    enviadas, os métodos que são chamados e como os objetos interagem entre si dentro
    de um determinado processo é o principal objetivo deste diagrama.
    
    \section{Diagrama de Classes}
    \begin{tikzpicture} 
      %\begin{umlpackage}{p}
        
        \begin{umlpackage}{Lista de Colecao} 
  
          \umlclass[x=-10, y=0]{Registro}{ 
          - colecao : Colecao[] \\
        }{
          + incluirColecao(Colecao: colecao): bool\\
          + EnviarListaColecao(): String\\
          + Excluir(String: id): bool\\
          + Alterar(id: String, nome: String, descricao: String):bool\\
        }
        
      \umlclass[x=0, y=-5]{Colecao}{
        - nome : String \\
        - descricao : String \\
        - id : String \\}{
        + Colecao(nome: String, descricoa: String)\\
        + getNome(): String\\
        + setNome(nome: String): void\\
        + getDescricao(): String\\
        + setDecriscao(descricao: String): void\\
        + getId(): String\\
        + setId(id: String): void\\
        + toString(): String\\}
      
      \umlclass[x=-10,y=-5]{Routes}{
        }{
          Salvar(): bool\\
          Listar(): String\\
          Alterar(id: String, nome: String, descricao: String): boll\\
          Excluir(id: String): boll\\
          }

      \umlclass[x=-10, y=-10]{Index}{}{}
            
          \end{umlpackage} 
          \umlcompo{Registro}{Colecao}
          \umlassoc{Registro}{Colecao}
  \end{tikzpicture}\\[0.5cm]
  O diagrama de classes é considerado por muitos autores como o mais importante e
  o mais utilizado diagrama da UML. Seu principal enfoque está em permitir a visualização
  das classes que irão compor o sistema com seus respectivos atributos e métodos,
  bem como em demonstrar como as classes do sistema se relacionam, se complementam e
  transmitem informações entre si. Este diagrama apresenta uma visão estática de como
  as classes estão organizadas, preocupando-se em definir a estrutura lógica das mesmas.

  \section{Conclusão}
  A maior dificuldade estava em fazer um Diagrama e só na implementação do próximo
  perceber que faltava uma funcionalidade ou classe no sistema. Isso poderia desestruturar
  o sistema mudando o pensamento lógico de algumas partes.\\
  Além disso, tivemos dificuldade na implementação do  Diagrama de Sequência,
  uma vez que não encontramos artigos relacionados que abrangia o tema.
  Desse modo, percebemos a complexidade de abstrair um problema real,
  e implementação de tal projeto.

  \newpage
  \begin{thebibliography}{4}
    \bibitem{DEVMEDIA}DEVMEDIA.\textbf{O Que é UML e Diagramas de Caso de Uso}: Introdução Prática à UML. 
    Recuperado em 17 de outubro de 2020,
    https://www.devmedia.com.br/o-que-e-uml-e-diagramas-de-caso-de-uso-introducao-pratica-a-uml/23408

    \bibitem{DEVMEDIA}DEVMEDIA.\textbf{Artigo SQL Magazine 64 - Utilizando UML}. 
    Recuperado em 17 de outubro de 2020,
    https://www.devmedia.com.br/artigo-sql-magazine-64-utilizando-uml/12665
  
    \bibitem{DEVMEDIA}DEVMEDIA.\textbf{Diagrama de Classes UML}. 
    Recuperado em 17 de outubro de 2020,
    https://www.devmedia.com.br/diagrama-de-classes-uml/12251
    \end{thebibliography}
     
  \end{document}